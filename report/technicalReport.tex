\documentclass[letterpaper, 10 pt, conference]{ieeeconf}  % Comment this line out
\IEEEoverridecommandlockouts                              % This command is only
% needed if you want to
% use the \thanks command
\overrideIEEEmargins
\usepackage[ruled,lined,linesnumbered]{algorithm2e}
\newtheorem{remark}{Remark}[section]


\newcommand{\bs}{\boldsymbol}
\usepackage{graphicx}
\usepackage[font=small,labelfont=bf]{caption}
\usepackage{subcaption}

% \IEEEoverridecommandlockouts                              % This command is only
% \overrideIEEEmargins
\usepackage{amssymb,amsmath}
% \usepackage{mathtools}
\usepackage{hyperref}
\newtheorem{Lemma}{Lemma}

% correct bad hyphenation here
\hyphenation{op-tical net-works semi-conduc-tor}

\begin{document}
\title{\LARGE \bf
Real-time dual-arm ping-pong ball juggling robot simulation
}
\author{
  Yuquan Wang, 
  \thanks{
    e-mail: \tt{ $\{$yuquan$\}$@kth.se}}
}

\maketitle
\thispagestyle{empty}
\pagestyle{empty}

\begin{abstract}
hwerd 
\end{abstract}

% \begin{keywords}
%   parallel and branching structure; virtual kinematic chain; constraint-based programming;
% \end{keywords}
\IEEEpeerreviewmaketitle

% ===============================================================================
\section{Introduction and related work}
\label{sec:intro}


according to the early pioneers \cite{rizzi1992distributed}, we should model the
striking process with proper flight dynamics of the ball and the impact dynamics
between the ball and the racket. 

The dual-arm coordination depending on the prediction of the ball, which could be used to design a feedforward controller. 
   
We prioritize the three juggling tasks using the cloest distance between the ball and the racket. 

\section{Problem formulation}

\subsection{Flight dynamics }
point mass under gravity force:

state space model: 
\begin{equation}
\label{eq:ss_ball_dynamics}
\end{equation}
We can construct an oberver of the flighting ball dynamics by feeding the discrete version of \eqref{eq:ss_ball_dynamics} as a state space model into a Kalman filter.  

\subsection{Impact dynamics}

\section{Proposed solution}


mirror algorithm \cite{buehler1994planning}


From the principal components in three-ball cascade juggling \cite{post2000principal}, we know that 

While the increasing randamness could reduce the controllability of a system, it may also be useful to avoid staying at locally optimal solutions.

\bibliographystyle{IEEEtran}
\bibliography{ref}
\end{document}

